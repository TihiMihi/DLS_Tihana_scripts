
\thispagestyle{empty}
\par\noindent                                           % Centre Title, and name

\begin{singlespace}
	\vspace{1cm}
	\begin{center}
		\Huge\bf\sc Absorption Correction using X-ray Tomography in Long-wavelength Macromolecular Crystallography \\
		\vspace{2mm}
		\large\bf Chemical Physics 5P Thesis
	\end{center}
	\vspace{1cm}
	\begin{center}
		\includegraphics[width=70mm]{UoECentred.eps}
	\end{center}
	\vspace{1.8cm}
	\begin{center}
		\bf\Large Tihana Stefanic\\                 % Replace with your name
		April 2024                         % Submission Date
	\end{center}
	
	%\vspace*{1.5cm}
	\begin{center}
		\small\itshape
		Submitted in partial fulfilment of the requirements \\  for the degree of Master of Chemical Physics (Hons) \\
		to the \\
		School of Chemistry \\
		College of Science and Engineering \\
		University of Edinburgh
	\end{center}
	
	
	\vspace*{5mm}   

	
	\vspace*{3cm}
	
	
	\vfill
	{\bf Total combined wordcount:} \\
	{\bf Supervisors:} Dr Ramona Duman, Dr Armin Wagner, and Professor Caroline Kirk              % Change to suit
\end{singlespace}

\newpage
%                                               Through page and setup 
%                                               fancy headings
%\setcounter{page}{1}                            % Set page number to 1
\footruleheight{1pt}
\headruleheight{1pt}
\lfoot{\small School of Chemistry}
\lhead{Chemical Physics 5P Thesis}
\rhead{\thepage}
\cfoot{}
\rfoot{1\textsuperscript{st} April, 2024 }

%

\newpage
\setcounter{page}{1}
\section*{Declaration}
The contents of this report have been approved by the supervisor running this project and I declare the work as a non-confidential project.\\\\
I declare that this thesis was composed by myself, that the work contained herein is my own except where explicitly stated otherwise in the text, and that this work has not been submitted for any other degree or qualification. \\
\bigskip


\begin{flushright}
	\begin{tabular}{m{5cm}}
		\\ \hline
		\centering Tihana Stefanic \\
		
	\end{tabular}\\
	
	\raggedleft\textit{Oxford, April 2024}
\end{flushright}
\newpage

\section*{Acknowledgements}

I would like to acknowledge and give my sincerest gratitude to my supervisors, Ramona and Caroline, for their unwavering support throughout this project.

I thank Ramona for her time and patience in teaching me everything I know about this project, and for providing the knowledge and insight crucial to its success.

I thank Caroline for her time in reading over the draft and providing helpful feedback throughout the year.

I would like to thank Christian Orr for his consistent support in answering questions about beamline I23 and otherwise, debugging my codes, laser-shaping, and for training me in too many things to list here. %crystal-fishing, sample loading, laser-shaping, and

I am very grateful to my friends, Patrick and Ismay, for their companionship and academic support throughout these experiments.
\newpage

\newpage

\printacronyms[template=description]
\newpage
\begin{abstract}
	Beamline I23 is a synchrotron instrument specifically designed for the study of long wavelength macromolecular crystallography experiments. The instrument operates in the low-energy wavelength range 1.5-5 Å, enabling the detection of X-ray absorption edges of biologically relevant light elements often found in proteins and enzymes. Access to these absorption edges is of particular importance to macromolecular crystallography as they provide crucial information for determining the structures and ions bound to macromolecules in the form of anomalous scattering.
    While I23 is a specialised beamline for operating in-vacuum to provide high signal-to-noise ratios even at the longest wavelengths, low-energy diffraction is still hindered by drastic absorption effects that must be accounted for. In response, I23 operates an X-ray tomography camera to reconstruct a 3D model of samples, allowing for the calculation of X-ray path lengths to determine analytical absorption correction factors.
    This project explores the effects of tomography-based reconstructions for measuring absorption coefficients and its applications in further experiments at I23. This approach was also combined with laser-shaping to assess the effects of combining two independent techniques of absorption corrections.
\end{abstract}
\newpage
\tableofcontents                                % Makes Table of Contents
\newpage


\listoffigures
\listoftables
\newpage