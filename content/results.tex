\newpage
\section{Results and Discussion}

The scope of experiments in this project involve extensive use of AnACor, the novel programme used at beamline I23 for adding an analytical correction to reflection intensities \cite{Lu2024}. Previous work from I23 reported on automatic segmentation for tomographic reconstruction \cite{Kazantsev2021}, but did not go further to investigate analytical absorption factors; AnACor has only recently become the routine for this.

A recent publication from I23, Lu \textit{et al.} \cite{Lu2024}, introduces the successful application of analytical absorption corrections based on the 3D-reconstructions from X-ray tomography implemented in AnACor. This paper describes two long-wavelength experiments performed on proteins of chlorite dismutase (Cld) and the membrane protein OmpK36 GD; their results indicating that this approach substantially improves data quality compared to the standard scaling practice based on \ac{sh}.

As of yet, Cld and Ompk, which crystallise in the triclinic \textit{P}1 and monoclinic \textit{C}2 space groups respectively, are the only samples to be published with analytical corrections calculated by AnACor in this way. Therefore, while the results are encouraging for long-wavelength \ac{mx}, knowledge of the suitability of analytical corrections is currently limited to very low-symmetry crystals.

For this reason, a majority of experiments in this work specifically investigate the effects of analytical corrections on higher-symmetry space groups, such as \textit{P}6\textsubscript{1}22, \textit{P}4\textsubscript{3}2\textsubscript{1}2, and \textit{I}2\textsubscript{1}3.

Since AnACor is still in development, reoccurring errors occasionally limit the ability to complete data processing and the consistency of results. Validation of the outputted results is therefore still necessary at every stage, and further steps are taken when possible to improve reliability.

When samples containing solely the mother-liquor that crystals were grown in were available, tomography data would be collected on these as well as the crystal samples. This is an example of an optional step taken to improve the reliability of calculating liquor absorption coefficients. The segmentation of loops containing only liquor allows for more reliable thresholds to be applied as there are fewer materials to account for, and therefore the liquor absorption coefficient is expected to be more precise.

Since low-energy tomographic scans are prone to high contrast artefacts which limit visibility in the segmentation, tomography scans, that follow diffraction experiments collected at or below 3 \unit{keV}, would often be collected at 3.5 \unit{keV} as well as the lower energy. Distinguishing between materials in the 3.5 \unit{keV} projections is somewhat easier, and therefore provides better segmentation for reflection data collected at lower energies.



%Finally ... preliminary tests

\subsection{Preliminary tests on AnACor}

The final reflection data heavily relies on the accuracy of liquor absorption coefficients; in response, the first experiments conducted for this project were a validation of the accuracy of AnACor's \textit{pre-process} calculations. Theoretical absorption coefficients are given by the reciprocal of the attenuation length, \textit{i.e.}  the distance over which the X-ray beam is absorbed, which can be calculated for a material from its chemical formula, density, and the energy in question. The first experiment on AnACor used cryo-cooled liquids of ethylene glycol (\ce{(CH2OH)2}) and santovac, a five ring polyphenyl ether (\ce{C30H22O4}), to compare experimentally calculated absorption coefficients from the programme with theoretical coefficients.

% Please add the following required packages to your document preamble:
% \usepackage{booktabs}
% \usepackage{graphicx}
\begin{table}[h]
\resizebox{\textwidth}{!}{%
\begin{tabular}{@{}ccccc@{}}
\toprule
\multicolumn{1}{l}{Energy (eV)} & \multicolumn{1}{l}{Attenuation length (µm)} & \multicolumn{1}{l}{Theoretical coefficient (1/µm)} & \multicolumn{1}{l}{Experimental coefficient (1/µm)} & \multicolumn{1}{l}{Percentage Difference (\%)} \\ \midrule
3000                            & 79.9406                                     & 0.012509                                           & 0.012413                                            & -0.774                                        \\
3500                            & 126.797                                     & 0.007887                                           & 0.007898                                            & 0.142                                       \\
4000                            & 189.989                                     & 0.005263                                           & 0.005335                                            & 1.35                                        \\
4500                            & 272.311                                     & 0.003672                                           & 0.003727                                            & 1.48                                       \\
5000                            & 376.408                                     & 0.002657                                           & 0.002741                                            & 3.16                                       \\
6000                            & 661.164                                     & 0.001512                                           & 0.001617                                            & 6.94                                        \\ \bottomrule
\end{tabular}%
}
\caption{Experimentally calculated santovac absorption coefficients from AnACor. Attenuation length calculated based on chemical formula \ce{C30H22O4} and density 1.195 \unit{\gram\per\cubic\cm} on the Center for X-Ray Optics \href{https://henke.lbl.gov/optical_constants/atten2.html}{X-ray Attenuation Length} platform.}
\label{santovac_table}
\end{table}


% Please add the following required packages to your document preamble:
% \usepackage{booktabs}
% \usepackage{graphicx}
\begin{table}[h]
\resizebox{\textwidth}{!}{%
\begin{tabular}{@{}lllll@{}}
\toprule
Energy (eV) & Attenuation length (µm) & Theoretical coefficient (1/µm) & Experimental coefficient (1/µm) & Percentage Difference (\%) \\ \midrule
3000 & 60.3724 & 0.016564 & 0.01615  & -2.49  \\
3500 & 95.0695 & 0.010519 & 0.010281 & -2.26 \\
4000 & 141.783 & 0.007053 & 0.00696  & -1.32 \\
4500 & 202.56  & 0.004937 & 0.00491  & -0.538 \\
5000 & 279.179 & 0.003582 & 0.003625 & 1.22 \\
6000 & 488.075 & 0.002049 & 0.002179 & 6.35 \\ \bottomrule
\end{tabular}%
}
\caption{Experimentally calculated ethylene glycol absorption coefficients from AnACor. Attenuation length calculated based on chemical formula \ce{C2H6O2} and density 1.1132
\unit{\gram\per\cubic\cm} on the Center for X-Ray Optics \href{https://henke.lbl.gov/optical_constants/atten2.html}{X-ray Attenuation Length} platform.}
\label{ethylene_table}
\end{table}

The experimental coefficients of santovac, published in \cref{santovac_table}, are within 1.5 \% of the theoretical value for energies between 3 and 4.5 \unit{keV}, while the experimental coefficients of ethylene glycol diverge from the theoretical values slightly more on average. Differences of 2-2.5 \% are generally considered sufficiently accurate for these experiments. It is therefore clear that the coefficients produced across the majority of energies by AnACor was successful.

In the case of both liquids, the percentage difference grows with energy, with more than a 6 \% difference at 6 \unit{keV}. %A similar trend is seen in \cref{ethylene_table}, with the sudden jump from 1.2 \% at 5 \unit{keV} to 6.3 \% for ethylene glycol at 6 \unit{keV}.
This trend indicates that AnACor overestimates the coefficients at 6 \unit{keV}. The explanation for this is that the higher energy data suffers from lower signal-to-noise ratios in the flat-field-corrected images, as higher energies result in lower intensities in the tomography experiments.

%Assessing from the results of these two proteins, AnACor appears to overestimate coefficients at higher energies and similarly underestimates coefficients at lower energies. A possible explanation for the latter is the effect of high contrast at low energies limiting distinctions between materials; this however does not explain the overestimates at higher energies. It is also possible that the calculations do not sufficiently account for the changing dependence of absorption on wavelength, which is less strong at higher energies and would explain the the discrepancies.

The majority of this study is performed at energies at or below 3.5 \unit{keV}, where discrepancies between these values was relatively low. It was therefore assumed that AnACor is sufficiently reliable for calculating absorption factors in subsequent experiments.% enough throughout these experiments. 

A second validation experiment on AnACor involved samples of chlorite dismutase (Cld) and the membrane protein OmpK36 GD. Data on these two samples has already been analysed in Lu \textit{et al.} \cite{Lu2024}, comparing the same approaches for absorption corrections discussed in this thesis. %: a \ac{sh} correction, analytical, and a combination of the two. 
The aim of the experiment described here, was to determine the accuracy of absorption coefficients of crystal and liquor published in the aforementioned paper, presented in \cref{Cld_Ompk_ACs} in the appendix. This was assessed with the assumption that the accuracy of absorption coefficients would correlate to the best possible data quality, where the intensity to uncertainty ratios ($I / \sigma$) peak and the $R_{merge}$ factors are diminished. %These parameters are referred to as the \textit{merging statistics}.

Taking the coefficients shown in \cref{Cld_Ompk_ACs} as a reference, the coefficients of crystal and of liquor were varied in intervals of \pm 10, while the remaining coefficients remained fixed. The crystal coefficients were varied from -20 \% to + 20 \%, and the liquor coefficients were varied from -20 \% up to + 30 \%; with every possible combination, this gave a total of 30 possible AnACor experiments. By setting up a variety of possible combinations of coefficient pairs, this experiment attempted to elucidate the optimal coefficients of both materials for improving merging statistics.%testing coefficient variations. %The crystal reference was

The merging statics for all 30 AnACor experiments are shown in the appendix in \cref{fig:cld_stats} for Cld and in \cref{fig:ompk_stats} for Ompk. The merging statistics of Cld tell a clearer story than that of Ompk: in \cref{fig:cld_stats} the $I/\sigma(I)$ ratios peak between +10 and +20\% in the crystal factor, and +10\% for liquor; looking at the $R$ factors, these values are minimised at the same variation. This indicates to the combination of coefficients that is presumably the most accurate for these materials, as it provides the best merging statistics.

While the differences in the maximum $I/\sigma(I)$ and minimum $R$ factors to their respective values at $\pm0 \%,\pm0$ \% are small, in a perfect experiment these optimal statistics would align at $\pm0,\pm0$, as this would indicate that the references AnACor calculated are very good estimates. %The underestimation here could be caused by several factors, including poor segmentation or insufficient threshold selections in pre-processing. Nonetheless, this procedure is still useful in correcting for small errors that prevent %deter 
%AnACor from obtaining the true values. 

The Ompk results, shown in \cref{fig:ompk_stats}, are less clear by comparison. While there is a boundary where $I/\sigma(I)$ is maximised, the merging $R$-factors do not change smoothly between experiments. Instead, there are only three values of the $R$-factor across all 30 runs with no incremental changes in between. %A smoothly changing R factor would indicate that the experiments This behaviour limits our ability to determine the optimal coefficients of liquor and ompk crystal. 
The absence of smoothly-changing behaviour between experiments is a limitation for finding the optimal absorption coefficients, as the optimal $R$ factor cannot be corroborated.%the R-factor cannot corroborate the optimal $I/\sigma(I)$.

%Establishing X-ray tomography as a valid approach: references for absorption correction calculations

%3D plots of finding the most optimal ACs for crystal and liquor using AnACor's AC values at 0.5 acceptance percentage as the reference

%This validation experiment ultimately took AC values calculated by AnACor in its pre-processing stage as a reference to run the post-processing of AnACor on a combination of AC of crystal and liquor varied in several increments from the reference. The aim of this was a check of AnACor's accuracy in determining the optimal AC that maximise the $I/ \sigma$ results while minimising the $R_{merge}$ factor.

Despite the discrepancy between the absorption coefficients used in this paper for Cld and the best possible statistics, the references used in this experiment were calculated in an earlier version of AnACor that came prior to several programme updates that improved the material threshold selection.

In the interest of time, this check was not applied to the crystals investigated in the remainder of this project. Nonetheless, the experiment is a useful validation for finding the optimal absorption coefficients from AnACor for merging statistics. %The scripts for this experiment have been complied to allow for users at I23 to apply this procedure to future tomography data. %A template of the scripts can be found at GitHub.


\subsection{Comparison of a spherical harmonics, analytical, and coupled approach} % empirical ?
% to absorption corrections
The primary focus of this project was to investigate the potential improvements in data quality by applying an analytical correction to diffraction data.
%Merging statistics and anomalous peak heights \cite{Lu2024}
In the interest of observing the effects of high absorption, data on crystal samples was largely collected at 3.0 and 3.5 \unit{keV}, some of the lowest energies accessible to I23.

The first sample to be used for this investigation was a test crystal of thaumatin (\ce{C19H29N3O3S}), collected at 3.0, 3.5, 4.0, and 4.5 \unit{keV}. Because this early experiment was used primarily for segmentation practice, the results for the merging statistics are presented in the appendix in \cref{fig:thaum1_stats}. Despite the segmentation being a rough model, the analytical approaches already showed an improvement compared to standard absorption correction protocols based on \ac{sh}, with the two datasets at lower energies favouring the coupled \ac{acsh} approach, and the remaining datasets favouring \ac{ac} without \ac{sh}. While these results were encouraging for the usefulness of AnACor, it is unclear why \ac{ac} performs better at higher energies. 

Subsequent tomography experiments were performed on two protein crystals of thermolysin, a thermostable metalloproteinase. The first crystal, referred to as Thermolysin 1, was collected at three energies: 3.0, 3.5, and 3.8 \unit{keV}, with datasets of varying orientations of $\kappa$ and $\phi$ at each energy, as listed in \cref{diffration_table} in the appendix. The Thermolysin 2 crystal was collected at 3.0 and 3.5 \unit{keV} with the $\kappa$ and $\phi$ orientations specified in the same table.

The merging statistics for individual datasets of Thermolysin 1 are shown in \cref{fig:tlys_9_3p0,fig:tlys_9_3p5,fig:tlys_9_3p8} in the appendix. Since there were no visible outliers, datasets for each energy were subsequently merged to produce the merging statistics illustrated in \cref{fig:tlys_9_stats}.%presenting outliers

\begin{figure}
    \centering
    \includegraphics{plots/exp1/tlys_9_P6122/merged_stats.pdf}
    \caption{Merging statistics for merged reflection data of Thermolysin 1 processed in \textit{P}6\textsubscript{1}22. Ratio of signal intensities to their uncertainties shown on the left; Overall $R_{merge}$ factor shown on the right.}
    \label{fig:tlys_9_stats}
\end{figure}


Taking into account both the increase in $I/\sigma(I)$ and decrease in $R_{merge}$, the merging statistics from Thermolysin 1 showed that \ac{acsh}, which combines \ac{ac} with spherical harmonics, was on average the best approach. While this was the case for every dataset of Thermolysin 1, some instances showed only a small improvement compared to \ac{sh}. Likewise, the standalone analytical approach was on average worse than \ac{sh} as well as \ac{acsh}.

Changes in energy are not expected to affect the reflection data; however, datasets that were collected subsequently are more susceptible to radiation damage from longer exposure to the beam. If a dataset exhibits signs of noticeable radiation damage, it is omitted from subsequently merged data.% If segmentation was then performed on a dataset collected early on in the tomography experiment, there can be discrepancies between the analytical model and a more-so radiation-damaged dataset.

%Following collection of reflection data, the AnACor results are used for structure refinement in Dimple, which generates anomalous density peaks for all the detected anomalous atoms and molecules based on the provided model. The results of Thermolysin 1

The merging statistics for individual datasets of Thermolysin 2 are shown in the appendix in \cref{fig:tlys_2_3p0,fig:tlys_2_3p5}. An immediate takeaway from these results is that the data quality of the second thermolysin is overall significantly better than the first, with on average higher reflection intensities and lower $R$-factors. Since there were no clear outliers, all datasets with common energies were once again merged to produce the results shown in \cref{fig:tlys_2_stats}. Both parameters clearly show the same trend, with \ac{acsh} being the dominating correction method across all datasets and energies. This is visible both in the $I/\sigma(I)$ ratios and the merging $R$ factors.
%that \ac{acsh} is once again the best correction method
%Results from Thermolysin 2 overwhelmingly show that \ac{acsh} is the dominating correction method, with $I/ \sigma$ highest and the R factors dipping across every dataset.

\begin{figure}
    \centering
    \includegraphics{plots/exp1/tlys_2_P6122/merged_stats.pdf}
    \caption{Merging statistics for merged reflection data of Thermolysin 2 processed in \textit{P}6\textsubscript{1}22. Ratio of signal intensities to their uncertainties shown on the left; Overall $R_{merge}$ factor shown on the right.}
    \label{fig:tlys_2_stats}
\end{figure}


The reflection data from DIALS and \textit{AnACor} were used for structural refinement in \textit{Dimple} to generate anomalous density maps for all the detected anomalous groups, based on the provided model in \ac{pdb} format. In thermolysin, the relevant anomalous groups are two methionine groups in positions 120 and 205 of the polypeptide chain of amino acid residues, %\cite{CLAUDIO1984}
and a zinc atom. The results for individual datasets of Thermolysin 1 can be found in \cref{fig:tlys9_met_peaks_3p0,fig:tlys9_met_peaks_3p5,fig:tlys9_met_peaks_3p8,fig:tlys9_zn_peaks_3p0_3p5,fig:tlys9_zn_peaks_3p8}, and for Thermolysin 2 in \cref{fig:tlys2_met_peaks_3p0,fig:tlys2_met_peaks_3p5,fig:tlys2_zn_peaks} in the appendix. % and \cref{fig:tlys2_met_peaks_3p5}.
While it is not common to all \ac{pdb}s of thermolysin, the model used in these refinements included a sulphate group, \ce{SO4}. The four highest anomalous density, or \textit{Anode} peaks, for merged reflection data from Thermolysin 1 are shown in \cref{fig:tlys_9_merged_peaks} and in \cref{fig:tlys_2_merged_peaks} for Thermolysin 2.


While the majority of individual datasets seen in \cref{fig:tlys9_met_peaks_3p0,fig:tlys9_met_peaks_3p5,fig:tlys9_met_peaks_3p8,fig:tlys9_zn_peaks_3p0_3p5,fig:tlys9_zn_peaks_3p8,fig:tlys2_met_peaks_3p0,fig:tlys2_met_peaks_3p5,fig:tlys2_zn_peaks} exhibit the highest anomalous signal when processed with \ac{acsh}, indicating to a boost from \ac{acsh} in anomalous density peaks, the improvements were reasonably small. Interestingly, the anomalous peaks measured from subsequently merged reflection data showed even smaller changes between correction methods, with no clear indication that \ac{acsh} leads to marginal benefits.

A clear difference here compared to the merging statistics is the lack of change in peak heights between methods. While there is a slight improvement in some cases, the peak heights remain overall unchanged regardless of the correction applied. This goes to show that despite the quality of reflections being clearly better when corrected with \ac{acsh}, analytical corrections are not sufficient for marginally improving anomalous slopes in thermolysin.%\textit{Dimple}.

\begin{figure}[h]
    \centering
    \includegraphics[width = 1.0 \textwidth]{plots/exp1/tlys_9_P6122/peaks/merged_peaks_so4.pdf}
    \caption{Top four \textit{Anode} peaks of merged Thermolysin 1. (Left to right): zinc, methionine (position 205), methionine (position 120), sulphate.}
    \label{fig:tlys_9_merged_peaks}
\end{figure}

\begin{figure}[h]
    \centering
    \includegraphics[width = 0.9 \textwidth]{plots/exp1/tlys_2_P6122/peaks/merged_peaks.pdf}
    \caption{Top four \textit{Anode} peaks of merged Thermolysin 2. (Left to right): zinc, methionine (position 205), methionine (position 120), sulphate.}
    \label{fig:tlys_2_merged_peaks}
\end{figure}
%While the heights of anomalous peaks are not a direct indicator of precision or quality, we can once again see that the coupled approach amplifies the detected anomalous signal from these groups.


Reflection data from these experiments was processed in \textit{P}6\textsubscript{1}22, the symmetry space group that thermolysin crystallises in. While processing in \textit{P}6\textsubscript{1}22 is therefore needed for obtaining correct reflections, high symmetry crystals can still be processed in low symmetry groups, such as \textit{P}\textsubscript{1}, to observe the general merging statistics of the experiment, independent of reflections.
Processing in low symmetry, however, reduces data multiplicity. Datasets across the same energy processed in \textit{P}\textsubscript{1} were merged to increase multiplicity, and the results of processing Thermolysin 1 and Thermolysin 2 in \textit{P}\textsubscript{1} are shown in \cref{fig:tlys_9_p1,fig:tlys_2_p1} in the appendix. With only a small fall in data quality, both results repeat the same trend seen for the merging statistics in the case of processing in \textit{P}6\textsubscript{1}22, where \ac{acsh} is on average the best correction across both parameters, with Thermolysin 2 having better statistics to begin with and a more significant boost in quality compared to Thermolysin 1.

%Tomography data
\subsection{Analytically-corrected data in $f"$-refinement for ion identification}

%As of this year, I23 has established a new method of ion identification using anomalous scattering known as f" refinement. This protocol uses a programme known as Phenix which has a feature for estimating the f' and f" values of a given element based on the (mtz and pdb) of the collected crystal data. The f"-refinement is becoming a routinely-used feature of ion identification due to the long wavelength capabilities of the beamline. 

While the anomalous slopes from \textit{Dimple} did not indicate to significant benefits in applying analytical corrections to thermolysin, the enhanced data quality can still be beneficial in other applications of long-wavelength diffraction data. This was tested in an ion identification experiment with the \ac{mx} software \textit{Phenix}, which has an extended platform known as \textit{phenix.refine} for the refinement of $f'$ and $f"$ parameters.

Because the merging statistics were especially satisfying for Thermolysin 2, reflection data from the first datasets of each energy produced 
%produced across the three correction methods 
in the prior experiment were used in \textit{phenix.refine} to test if analytical corrections improve the refinement of $f"$. The experiment tested for the detection of sulphur and zinc atoms at 3.0 and 3.5 \unit{keV}, by refining the experimental $f"$ of the specified atom according to the theoretical $f'$. Each refinement ran for five cycles, and the input parameters and final results are shown in \cref{phenix_table} in Appendix A2. The percentage differences between theoretical and experimentally determined $f"$ values are illustrated in \cref{fig:phenix_plot}.%Phenix is a comprehensive software package for \ac{mx} structure determination; this particular platform allows for the refinement of absorption parameters $f'$ and $f"$ based on the reflection data and \ac{pdb} structure provided.

%As described in the methodology, t
%The first stage of \textit{phenix.refine} uses the reflection data provided to produce a modified version of the molecular structure in \ac{pdb} format. This updated molecular structure is then used in the second stage with the same reflection data to estimate the theoretical $f'$ and $f"$ values of specified atoms that are believed to be in the structure. In the case of this experiment, $f'$ was fixed to its theoretical value for a given atom and energy, allowing for an experimental $f"$ to be refined. This value is then corroborated with the theoretical value to determine whether the atom type can be correctly identified.% test the ability of identifying atoms of sulphur and zinc in the crystal by determining their $f"$ peak.


\begin{figure}[h]
    \centering
    \includegraphics[width = 0.7\textwidth]{plots/exp1/tlys_2_P6122/FDP_percent_diff.pdf}
    \caption{Percentage difference with respect to theoretical $f"$ for atoms of sulphur and zinc in Thermolysin 2. Values obtained from \textit{Phenix.refine} in the Phenix software package for macromolecular structure determination.}
    \label{fig:phenix_plot}
\end{figure}

%Results from Phenix refine for the six sets of reflection data of Thermolysin 2 (three methods across 3.0 and 3.5 \unit{keV}) are shown in \cref{phenix_table} in Appendix A2 and are illustrated in \cref{fig:phenix_plot}.
The first observation from \cref{fig:phenix_plot} is that nearly every refinement underestimates the value of $f"$. It is also clear that refinement was generally underestimated less at 3.5 \unit{keV} than at 3.0 \unit{keV}, and identification of the sulphur edge was much more successful than that of zinc. Assuming the \ac{pdb} used adequately matches the residual atoms of thermolysin, this would suggest that identification of zinc by $f"$ refinement cannot be achieved at such long wavelengths, and while they may improve data quality, the analytical corrections cannot sufficiently compensate for it. %could indicate that the zinc atom detected by Dimple is in fact not zinc, or

For the sulphur atom, this is no longer the case. While standalone \ac{ac} fails to identify sulphur within an appropriate margin of error, comparison with \ac{sh} shows that \ac{acsh} brought the refined $f"$ value much closer to theoretical $f"$. At both energies, this approach successfully identifies the sulphur edge within 2 \% of the theoretical value. This result is quite meaningful, as it demonstrates a clear, beneficial application of X-ray tomography improvements in \ac{mx} experiments.%Another explanation, is the fact that the anomalous signal of sulphur is much stronger than that of zinc this reflection data was collected at 3.0 and 3.5 \unit{keV}, which are above the sulphur edge %which may indicate that the \ac{pdb} used did not sufficiently match the test crystals to correctly model the zinc atom.

Given that this experiment has only been performed on one crystal, with reflections from only one dataset per energy, it would be of interest to see whether this behaviour is reproduced in $f"$-refinement at long wavelengths with other crystals and atom types. %Another potential experiment would be to test if 

This method of ion identification through Phenix is very novel, with ongoing experiments at I23 still testing the optimal protocol for the procedure as well as its limitations in long-wavelength experiments.

\subsection{Analytical tomography corrections with laser-shaped samples}

Following experiments that showed the successes and limitations of tomographic reconstruction and analytical corrections, the next question to investigate was about its compatibility with another form of absorption correction. Laser-shaping has previously been used in cryogenic \ac{mx} to remove the non-diffracting material contributing to background noise \cite{Basu2019}, thus improving the diffraction quality and likewise the standard \ac{sh} absorption corrections that only assume for the diffracting material in a sample.

Both X-ray tomography and laser-shaping have proven to improve the quality of low-energy diffraction, but their combined effects had not yet been explored. In the following experiment, two crystals of insulin and two of proteinase K were used for a preliminary test of what this combined effect would look like. Each protein experiment had one control crystal and one that was laser-shaped to remove a majority of the loop and solvent. These samples were then segmented to produce the models shown in \cref{avizo_insulin} and \cref{avizo_proteinasek} in Appendix A1.

\begin{figure}[h]
    \centering
    \includegraphics{plots/exp2/ins_stats.pdf}
    \caption{Merging statistics for the laser-shaped and control crystals of insulin at 3.0 \unit{keV}. Ratio of signal intensities to their uncertainties shown on the left; Overall $R_{merge}$ factor shown on the right.}
    \label{fig:insulin}
\end{figure}

The laser-shaped samples were ablated to make a spherically shaped crystal that would ideally have been the only material entity. While the crystals were spherical in shape, the size of the crystals used limited the samples to still containing some mother liquor. The tip of laser-shaped samples were therefore not \textit{perfect} crystal spheres, but were entirely void of loop material.

%The merging statistics obtained from this experiment tell somewhat different stories, depending on the protein. 

\begin{figure}[h]
    \centering
    \includegraphics{plots/exp2/prot_stats.pdf}
    \caption{Merging statistics for the laser-shaped and control crystals of Proteinase K at 3.0 \unit{keV}. Ratio of signal intensities to their uncertainties shown on the left; Overall $R_{merge}$ factor shown on the right.}
    \label{fig:proteinasek}
\end{figure}

The insulin crystals, shown in \cref{fig:insulin}, suggest there is little to no change in the \ac{acsh} approach between the laser-shaped and control. \Ac{sh}, on the other hand, experiences a huge boost from laser-shaping. This is to be expected for multiple reasons; spherical harmonics assume a spherical shape of the crystal, and therefore model the effects of absorption more reliably with a spherically-shaped sample; likewise, with \ac{sh} unable to account for non-diffracting material in the sample, the removal of most thereby leads to better diffraction and better empirical models. This boost in \ac{sh} is also seen with proteinase K in \cref{fig:proteinasek}, though less significant.

A common feature to both proteins is that in the case of the laser-shaped sample, \ac{acsh} and \ac{sh} remain nearly identical across both parameters.
This suggests that, while tomography-based reconstruction and laser-shaping are both successful absorption corrections in their own right, the effect of combining these methods at long wavelengths remains unchanged.

An unexpected feature of the proteinase K results is the fall in \ac{ac} and \ac{acsh} from the control to laser-shaped sample. While there is less reason for analytical corrections to be improved by laser-shaping, as non-diffracting material is already accounted for in detail, it is unusual for the corrections to perform worse on a laser-shaped sample. Since this effect is not observed in the insulin crystals, it is necessary to repeat this experiment with a larger sample pool to determine if this behaviour is an outlier.%It is believed that this discrepancy can be attributed to the surface of laser-shaped insulin, seen in \cref{avizo_insulin}, which was covered by bubbles of what was presumed to be liquor. Aside from the fact that these bubbles contribute to background noise while not contributing to diffraction, the high contrast of this surface made detailed segmentation especially challenging, making . The limitations on the analytical model could therefore be the cause. % of this

While the results from insulin indicate that the effects of only laser-shaping and of only analytical corrections are comparable, the results from proteinase K tell a different story, suggesting that laser-shaping only benefits \ac{sh} while being in fact detrimental to analytical approaches. The control sample of proteinase K with \ac{acsh} likewise performs better than any corrections among either crystal. %there is a noticeable improvement in laser-shaping compared to standalone tomography. This is unexpected when compared to the trend seen with insulin. The control, however, does not follow the expected behaviour either; with \ac{sh} and \ac{acsh} being nearly identical among both parameters. This could be attributed simply to poor diffraction; however, more data must be analysed to reach a conclusive picture of what is happening.


%Previous studies done on tomography-based analytical corrections have focused only on the low-symmetry space groups \textit{P}\textsubscript{1}, and \textit{C}\textsubscript{2}, as it was believed that high symmetry crystals with a naturally high multiplicity would not benefit significantly from analytical approaches that depend on the sample geometry. Given that thermolysin, and insulin in particular, are both high-symmetry crystals crystallising in  \textit{P}6\textsubscript{1}22 and \textit{I}2\textsubscript{1}3 respectively, the preliminary results from these crystals would suggest that the benefits of analytical corrections are not limited to only low-symmetry space groups.

%This is particularly relevant for future applications of this technique in soft-X-ray experiments, but is not limited to long wavelength experiments. Analytical corrections are useful not only for long-wavelength \ac{mx}, but also for highly-absorbing crystal samples.

%Testing tomo on lysosyme to try to detect sodium anomalous signal

%Mention LASER-shaping already worked on identifying magnesium in topoisomerase crystals. Hypothesis: tomography can do the same

%\subsection{Detecting sodium in lysozyme with analytical tomography corrections}

%\begin{figure}
    %\centering
    %\includegraphics{plots/exp3/lys_stats_grid.pdf}
    %\caption{Lysozyme Datasets}
    %\label{fig:enter-label}
%\end{figure}

%\begin{figure}
    %\centering
    %\includegraphics{plots/exp3/merged_stats_grid.pdf}
    %\caption{Merged Lysozyme datasets}
    %\label{fig:enter-label}
%\end{figure}

\subsection{Errors and Improvement}

Faults in the absorption coefficients are particularly sensitive to the \textit{thresholding} in AnACor that predicts the location of a material. While the alignment of the threshold tends not to be a problem for segmented datasets, datasets with different flat fields that reuse the same segmentation model have recently been having minor displacement problems in the newest version of AnACor. Indications of coefficients being overestimated or underestimated, such as with Cld and Ompk, can be attributed to these problems.  

While it is unclear whether this affects the quality of results in this project, further steps must be taken to resolve the material selection and ensure analytical pathways are computed precisely.

Current efforts are being put forward to developing an automatic segmentation procedure at I23. This would significantly boost the speed of data processing jobs in tomography, as segmentation is the most meticulous aspect of the routine.

%**CK says: comment on the structural refinements, rather than just merging statistics... show improvements in the structural modelling? This would put the work more in context and allow for more discussion**

%Faults in anacor thresholding, particularly at processing datasets collected at multiple energies

%segmentation at high contrast

%There are several challenges associated with the segmentation of X-ray data of protein crystals.