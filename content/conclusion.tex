\section{Conclusion and Future Outlooks}

This project set out to establish X-ray tomography as a valid technique for calculating analytical absorption corrections and to investigate its applications further in long wavelength crystallography. The global parameters for merging statistics were examined for every experiment, comparing the standard empirical and analytical corrections individually and combined.

There is clear evidence to suggest that combining X-ray tomography with empirical calculation methods provides the best account of absorption effects. Further experiments using the analytically corrected reflection data in anomalous density refinement as well as $f"$-refinement showed promising effects, both in boosting the detectable anomalous signal and in identifying sulphur respectively.
Furthermore, we can see this effect carry with high-symmetry groups as well as low-symmetry groups; analytical models were previously thought to only be useful for low-symmetry crystals with low multiplicity.

While preliminary results suggest there is little benefit to applying tomography reconstruction to laser-shaped samples, the effects of this will be investigated further as there is currently limited evidence to conclude which correction is more beneficial.

There are two tomography experiments currently underway at I23 on determining the presence of unknown ions in protein crystals that were not detected by standard corrections. These will investigate detecting the presence of magnesium in a protein of topoisomerase, as well as detecting two atoms of sodium in lysozyme. These experiments will be insightful for further applications of X-ray tomography in detecting weaker anomalous signals from crystals of biological importance.
%Future experiment on determining the presence of magnesium in topoisomerase

%Repeating laser-shaping with larger crystals so that as close to 100 \% of sample exposed to beam is crystal entity -> test this on crystals where specific atoms are trying to get detected