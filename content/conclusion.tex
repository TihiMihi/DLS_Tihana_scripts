\section{Conclusion and Future Outlooks}

This project set out to establish X-ray tomography as a valid technique for calculating analytical absorption corrections and to investigate its applications further in long wavelength crystallography. The global parameters for merging statistics were examined for every experiment, comparing the standard empirical and analytical corrections individually and combined.

There is clear evidence to suggest that combining X-ray tomography with empirical calculation methods provides the best account of absorption effects, as was shown in previous studies by I23 \cite{Kazantsev2021,Lu2024}. Further experiments using the analytically corrected reflection data in anomalous density refinement as well as $f"$-refinement showed promising effects, both in boosting the detectable anomalous signal and in identifying sulphur respectively.
Furthermore, we can see this effect carry with high-symmetry groups in this project, in addition to low-symmetry groups in recently published work on AnACor \cite{Lu2024}.%; analytical models were previously thought to only be useful for low-symmetry crystals with low multiplicity.

Previous studies done on tomography-based analytical corrections have focused only on the low-symmetry space groups \textit{P}\textsubscript{1}, and \textit{C}\textsubscript{2}, as it was believed that high symmetry crystals with a naturally high multiplicity would not benefit significantly from analytical approaches that depend on the sample geometry. Given that thermolysin, and insulin in particular, are both high-symmetry crystals crystallising in \textit{P}6\textsubscript{1}22 and \textit{I}2\textsubscript{1}3 respectively, the preliminary results from these crystals presented in this thesis would suggest that the benefits of analytical corrections are not limited to only low-symmetry space groups.

This is particularly relevant for future applications of this technique in soft-X-ray experiments, but is not limited to long wavelength experiments. Analytical corrections are useful not only for long-wavelength \ac{mx}, but also for highly-absorbing crystal samples in other branches of crystallography.

While preliminary results from insulin and proteinase K suggest there is little benefit to applying tomography reconstruction to laser-shaped samples, the effects of this will be investigated further as there is currently limited evidence to conclude which correction is more beneficial for high and low symmetry space groups, and whether there is a pattern in what crystals benefit more from one kind of correction technique.

There are two tomography experiments currently underway at I23 on determining the presence of unknown ions in protein crystals that were not detected by standard corrections. These will investigate detecting the presence of magnesium in a protein of topoisomerase, as well as detecting two atoms of sodium in lysozyme. These experiments will be insightful for further applications of X-ray tomography in detecting weaker anomalous signals from crystals of biological importance.
%Future experiment on determining the presence of magnesium in topoisomerase

%Repeating laser-shaping with larger crystals so that as close to 100 \% of sample exposed to beam is crystal entity -> test this on crystals where specific atoms are trying to get detected